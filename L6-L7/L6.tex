\documentclass{article}

%\title{MAC0338 - Entrega da lista 6/7}
\title{MAC0338 - Lista 6}
\author{}
\date{}

\usepackage[a4paper, margin=1.25cm]{geometry}
\usepackage{amsmath, amssymb, amsthm}
\usepackage{bm}
\usepackage[skip=2pt]{parskip}    
\usepackage{wrapfig}
\usepackage{braket}
\usepackage{mathtools}
\usepackage[noend, linesnumbered]{algorithm2e}
\SetKwComment{Comment}{$\triangleright$\ }{}
\RestyleAlgo{ruled}

\newcommand{\N}{\mathbb{N}}
\newcommand{\bigO}[1]{\ensuremath{\mathcal{O}(#1)}}

\begin{document}

\maketitle

\section*{Exercício 2}
% enunciado
A entrada é uma sequência de números $x_1, x_2, \dots, x_n$ onde $n$ é par.
Projete um algoritmo que particione a entrada em $n/2$ pares da seguinte maneira. Para cada par, computamos a soma de seus números. Denote por $s_1, s_2, \dots , s_{n/2}$ as $n/2$ somas. O algoritmo deve encontrar uma partição que minimize o máximo das somas e deve ser O$(n \log n)$.

\bigskip

% resposta
\subsection*{Resposta:}

\begin{algorithm}
  \caption{MinimizaSomas(x[n])}
  \label{alg:MinimizaSomas}
  \SetKwProg{Fn}{Function}{:}{end}
  \Fn{MINIMIZA-SOMAS($x[n]$)}{
    ordena(x) \Comment*[r]{O(n log n)} 
    cria vetor vazio $s$ com índices de 1 até $n/2$ \Comment*[r]{O(n)}
    para $i \gets 1$ até inclusive $n/2$: \Comment*[r]{O(n/2)}
    \qquad $s_i \gets x[i] + x[n-i+1]$ \Comment*[r]{O(1)}
    \Return{s} \Comment*[r]{O(1)}
  }
\end{algorithm}

\subsubsection*{Corretude}
%https://cs.stackexchange.com/questions/124596/n-numbers-n-2-pairs-minimizing-the-maximum-sum-of-a-pairing-proving-greedy-al
Vamos provar a corretude do algoritmo acima.

Ordene $x_1 \leq  \dots \leq x_n$.

Caso $(x_1, x_n)$ forme um par, então já temos o que queremos.

Caso contrário, suponha que $(a_1,a_k)$ seja um par tal que $r \neq n$, e $(a_n,a_l)$ seja um par tal que $s \neq 1$. Para fazer a escolha gulosa, troque $a_k$ com $a_n$ e $a_l$ com $a_1$. Agora temos $(a_1,a_n)$ e $(a_k,a_l)$. Note que
\begin{itemize}
  \item dentre os pares antigos $(a_1,a_k)$ e $(a_n,a_l)$, $a_1+a_k \leq a_n+a_l$;
  \item a soma do novo par $a_1+a_n$ é menor ou igual a $a_n+a_l$, já que $a_1 \leq a_l$;
  \item a soma do novo par $a_k+a_l$ é menor ou igual a $a_n+a_l$, já que $a_k \leq a_n$.
\end{itemize}
Portanto, a escolha gulosa adotada não aumenta o máximo de nenhuma soma.

Agora, considere a sequência sem $x_1$ nem $x_n$, ou seja, $x_2, \dots, x_{n-1}$. Usando o mesmo raciocínio, os pares sempre serão da forma $(x_i,x_{n-i+1})$ e a soma máxima irá se manter mínima.

\subsubsection*{Consumo de tempo}
Note que, de acordo com as anotações no algoritmo \ref*{alg:MinimizaSomas}, o consumo de tempo é
\begin{align*}
  O(n\log n) + O(n) + (O(n/2) \cdot O(1)) + O(1) = O(n \log n) ,
\end{align*}
como queríamos.

\newpage

\section*{Exercício 4}
%enunciado
Seja $1,\dots, n$ um conjunto de tarefas. Cada tarefa consome um dia de trabalho; durante um dia de trabalho somente uma das tarefas pode ser executada. Os dias de trabalho são numerados de 1 a $n$. A cada tarefa $t$ está associado um prazo $p_t$
: a tarefa deveria ser executada em algum dia do intervalo $1 . . p_t$. A cada tarefa $t$ está associada uma multa $m_t \geq 0$. Se uma dada tarefa $t$ é executada depois do prazo $p_t$, sou obrigado a pagar a multa $m_t$ (mas a multa não depende do número de dias de atraso).

\textbf{Problema}: Programar as tarefas (ou seja, estabelecer uma bijeção entre as tarefas e os dias de trabalho) de modo a minimizar a multa total. Considere o algoritmo que primeiramente ordena as tarefas pela multa; a seguir, considera uma
a uma cada tarefa, escalonando-a no último dia livre dentro do seu prazo. Se não houver dia livre dentro do prazo, escalona a tarefa no último dia livre.

Prove que esse algoritmo está correto, ou seja, produz um escalonamento com multa mínima. Analise seu consumo de tempo.

\bigskip

%resposta
\subsection*{Resposta:}
Seja $O$ uma solução ótima.

Considere uma tarefa $t_i$ dentre as tarefas.
Se $t_i$ está escalonada em algum dia antes do próprio prazo, então podemos trocar $t_i$ com a tarefa $t_j$, escalonada no prazo de $t_i$, e obter uma solução $O'$ com multa igual a $O$.
Se $t_i$ está escalonada em algum dia depois do próprio prazo, então vamos separar em 2 casos:
\begin{itemize}
  \item Se $p_{t_i} \leq i$, então existe uma tarefa $t_h$ que possui multa menor ou igual a $t_i$. Logo, basta trocar $t_i$ com $t_h$ e obter uma solução $O'$ com multa menor que $O$. Mas como $O$ é ótima, isso não é possível.
  \item Se $p_{t_i} > i$, então podemos trocar $t_i$ com qualquer tarefa que esteja escalonada após o prazo de $t_i$ e obter uma solução $O'$ com multa menor ou igual a $O$.
\end{itemize}
Logo, o algoritmo guloso é ótimo.

\end{document}