\documentclass{article}

\title{MAC0338 - Lista 9}
\author{}
\date{}

\usepackage[a4paper, margin=1.25cm]{geometry}
\usepackage{amsmath, amssymb, amsthm}
\usepackage{bm}
\usepackage[skip=2pt]{parskip}    
\usepackage{wrapfig}
\usepackage{braket}
\usepackage{multicol}
\usepackage{mathtools}
\usepackage[noend, linesnumbered]{algorithm2e}
\SetKwComment{Comment}{$\triangleright$\ }{}
\RestyleAlgo{ruled}

\begin{document}

\maketitle

\section*{Exercício 3}
% enunciado
\textbf{(CLRS 17.2-1)} Uma sequência de operações sobre uma pilha é executada numa pilha cujo tamanho nunca excede $k$. Depois de cada $k$ operações, uma cópia da pilha toda é feita para propósito de back-up. Mostre que o custo de $n$ operações sobre a pilha, incluindo as operações de cópia para back-up, é $O(n)$, atribuindo valores adequados de créditos a cada operação.

\bigskip

% resposta
\subsection*{Resposta:}
Vamos analisar os custos reais de cada operação da pilha:
\begin{multicols}{4}
\begin{itemize}
  \item \textsc{push}: 1
  \item \textsc{pop}: 1
  \item \textsc{multipop}: $\min(k,s)$
  \item \textsc{backup}: $\min(k,s)$
\end{itemize}
\end{multicols}

Agora, vamos desenvolver os custos amortizados de cada operação da pilha:
\begin{multicols}{4}
\begin{itemize}
  \item \textsc{push}: 3
  \item \textsc{pop}: 0
  \item \textsc{multipop}: 0
  \item \textsc{backup}: 0
\end{itemize}
\end{multicols}
Suponha que a operação \textsc{backup} é uma varredura na pilha inteira (ou seja, uma consulta de cada elemento empilhado, mas sem realizar \textsc{pop}'s).
Cobramos 3 para cada operação de \textsc{push}: 1 para o custo do próprio \textsc{push}, 1 para compensar uma futura operação \textsc{pop} e 1 para compensar uma  cópia para o \textsc{backup}.
Como a pilha tem capacidade máxima de $k$ elementos e o \textsc{backup} é realizado a cada $k$ operações, então sempre temos crédito suficiente para, no pior caso, um \textsc{backup} da pilha inteira.
Portanto, asseguramos que a quantia que temos de crédito é sempre não-negativa.
Assim, para qualquer sequência de $n$ operações listadas, o custo amortizado total é um limite superior para o custo real.
Visto que o custo amortizado de uma operação é $O(1)$, então o custo amortizado total para uma sequência de $n$ operações é $n \cdot O(1) = O(n)$, que também é o custo real.

\newpage

\section*{Exercício 6}
% enunciado

\bigskip

% resposta
\subsection*{Resposta:}

\end{document}