\documentclass{article}

\title{MAC0338 - Entrega da lista 4}
\author{}
\date{}

\usepackage[a4paper, margin=1.25cm]{geometry}
\usepackage{amsmath, amssymb, amsthm}
\usepackage{bm}
\usepackage[skip=2pt]{parskip}    
\usepackage{wrapfig}
\usepackage{braket}
\usepackage{mathtools}
\usepackage[noend, linesnumbered]{algorithm2e}
\SetKwComment{Comment}{$\triangleright$\ }{}
\RestyleAlgo{ruled}

\newcommand{\N}{\mathbb{N}}
\newcommand{\bigO}[1]{\ensuremath{\mathcal{O}(#1)}}

\begin{document}

\maketitle

\section*{Exercício 8}
% enunciado
\textit{Prove que o \textsc{CountingSort} é estável.}

\bigskip

% resposta
Vamos provar que o counting sort é estável construtivamente, usando como referência o código \textsc{CountingSort(A,n)} da aula 7, slide 7.
Vamos focar nas linhas 7, 8 e 9:
\begin{algorithm}
  \caption{CountingSort (linhas 7, 8 e 9)}
  para $j \gets n$ decrescendo até 1 faça \\
  \qquad $B[C[A[j]]] \gets A[j]$ \\
  \qquad $C[A[j]] \gets C[A[j]]-1$
\end{algorithm}

Note que estamos percorrendo o vetor $A = [a_1, a_2, \dots, a_n]$ de trás para frente e preencheremos um vetor $B = [b_1, b_2, \dots, b_n]$.
Também temos um vetor $C = [c_1, c_2, \dots, c_k]$, que mostrará o índice que cada elemento do vetor $A$ ocupará no vetor ordenado $B$.

Seja $a_x = a_y = m$ tal que $x < y$, ou seja, $a_x$ está localizado antes de $a_y$ no vetor $A$. Como estamos percorrendo o vetor $A$ de trás para frente, então $a_y$ será inserido no vetor $B$ na posição $c_m$, e logo em seguida $c_m \gets c_m - 1$. Note que os valores de $C$ nunca crescem, apenas decrescem. Portanto, em seguida $a_x$ será inserido no vetor $B$ \textbf{pelo menos} na posição $c_m - 1$, que com certeza está localizado em uma posição anterior a $a_y$.

Logo, o algoritmo mantém a ordem relativa de elementos com chaves iguais durante o processo de ordenação. Isso prova que o counting sort é estável.

\newpage

\section*{Exercício 14}
% enunciado
\textit{Escreva um algoritmo que multiplica dois números complexos $a + bi$ e $c + di$ usando apenas três multiplicações reais. O seu algoritmo deve receber como entrada os números $a, b, c, d$ e devolver os números $ac - bd$ (componente real do produto) e $ad + bc$ (componente imaginária do produto).}

\bigskip

% resposta
\begin{algorithm}
  \caption{MultiplicaComplexos(a,b,c,d)}
  $x \gets a \cdot c$ \\
  $y \gets b \cdot d$ \\
  $z \gets (a + b) \cdot (c + d)$ \\
  $real \gets x - y$ \\
  $imaginario \gets z - x - y$ \\
  \Return{$real, imaginario$}
\end{algorithm}

Por definição, a parte real da multiplicação entre $a + bi$ e $c + di$ é $ac - bd$. Usamos uma multiplicação real para calcular $ac$ e outra para $bd$.

A parte imaginária é $ad + bc$. Agora, note que
\begin{align*}
  (a+b) \cdot (c+d) = ac + ad + bc + bd \implies ad + bc = (a+b) \cdot (c+d) - ac - bd ,
\end{align*}
e usamos mais uma multiplicação real para calcular $(a+b) \cdot (c+d)$.

Assim, calculamos a parte real e imaginária de uma multiplicação de complexos com 3 multiplicações reais.

\newpage

\section*{Exercício 15}
% enunciado
\textit{No \textsc{Select-BFPRT}, os elementos do vetor são divididos em grupos de 5. O algoritmo continua linear se dividirmos os elementos em grupos de 7? E em grupos de 3? Prove sua resposta.}

\bigskip

% resposta
Vamos analisar o consumo de tempo quando dividimos os elementos em grupos de 7.

Baseando-se no CLRS, o número de elementos maiores que $x$ (mediana das medianas) nesse caso seria
$$4\Bigl(\Big\lceil\frac{1}{2}\Big\lceil\frac{n}{2}\Big\rceil\Big\rceil-2\Bigr) \geq \frac{4n}{14} - 8 .$$
Assim, a função de recorrência seria, para uma constante $63$,
\begin{align*}
T(n) =
  \begin{cases*}
    O(1) &\text{se} $n < 63$ \\
    T(\lceil n/7 \rceil) + T(6n/14+8) + O(n) &\text{se} $n \geq 63$
  \end{cases*}
\end{align*}
Provaremos, assim como em CLRS, que $T(n) \leq cn$ para alguma constante $c$ adequadamente grande e para todo $n>0$.
Também escolhemos uma constante $a$ tal que $O(n) \geq an$ para todo $n>0$.
Logo, temos
\begin{align*}
  T(n) &\leq c \lceil\ n/7 \rceil + c(6n/14 + 8) + an \\
  &\leq cn/7 + c + 6cn/14 + 8c + an \\
  &= 8cn/14 + 9c + an \\
  &= cn + (-6cn/14 + 9c + an)
\end{align*}
que é no máximo $cn$ se $-6cn/14 + 9c + an \leq 0 \implies c \geq \frac{7a}{3}(n/(n-21))$ quando $n > 21$.
Para $n=63$, temos $n/(n-21) \leq 3$, sendo a constante $c \geq \frac{7a}{3} \cdot 3 = 7a$.
Logo, o tempo de execução é linear quando dividimos os elementos em grupos de 7.

\bigskip

Agora, vamos analisar o consumo de tempo quando dividimos os elementos em grupos de 3.

Analogamente, o número de elementos maiores que $x$ (mediana das medianas) nesse caso seria
$$2\Bigl(\Big\lceil\frac{1}{2}\Big\lceil\frac{n}{2}\Big\rceil\Big\rceil-2\Bigr) \geq \frac{2n}{9} - 2 .$$
Assim, a função de recorrência seria, para uma constante $l$,
\begin{align*}
T(n) =
  \begin{cases*}
    O(1) &\text{se} $n < l$ \\
    T(\lceil n/3 \rceil) + T(7n/9 + 2) + O(n) &\text{se} $n \geq l$
  \end{cases*}
\end{align*}
Provaremos, assim como em CLRS, que $T(n) \leq cn$ para alguma constante $c$ adequadamente grande e para todo $n>0$.
Também escolhemos uma constante $a$ tal que $O(n) \geq an$ para todo $n>0$.
Logo, temos
\begin{align*}
  T(n) &\leq c \lceil\ n/3 \rceil + c(7n/9 + 2) + an \\
  &\leq cn/3 + c + 7cn/9 + 2c + an \\
  &= 9cn/9 + 3c + an \\
  &= cn + (3c + an)
\end{align*}
que é no máximo $cn$ se $3c + an \leq 0 \implies c \leq -an/3$ para qualquer $n > 0$.
Escolhendo $n \geq 3$, temos $-n/3 \leq 1$, e então $c \geq a$.
Logo, o tempo de execução também é linear quando dividimos os elementos em grupos de 3.


\end{document}