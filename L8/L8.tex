
\documentclass{article}

\title{MAC0338 - Entrega da lista 8}
\author{}
\date{}

\usepackage[a4paper, margin=1.25cm]{geometry}
\usepackage{amsmath, amssymb, amsthm}
\usepackage{bm}
\usepackage[skip=2pt]{parskip}    
\usepackage{wrapfig}
\usepackage{braket}
\usepackage{mathtools}
\usepackage[noend, linesnumbered]{algorithm2e}
\SetKwComment{Comment}{$\triangleright$\ }{}
\RestyleAlgo{ruled}

\begin{document}

\maketitle

\section*{Exercício 3}
% enunciado
Uma fórmula booleana $\mathcal{C}$ sobre um conjunto $X$ de variáveis booleanas (não necessariamente em CNF) é uma tautologia se toda atribuição a $X$ satisfaz $\mathcal{C}$. O problema tautologia consiste em, dado $X$ e $\mathcal{C}$, decidir se $\mathcal{C}$ é ou não uma tautologia. Prove que o problema tautologia está em coNP.

\bigskip

% resposta
% https://cs.stackexchange.com/questions/76579/proof-that-taut-is-conp-complete-or-that-a-problem-is-conp-complete-if-its-comp
\subsection*{Resposta:}
Vamos provar que o problema \textsc{TAUT} (tautologia) está em coNP.
Podemos provar, equivalentemente, que o complemento de \textsc{TAUT} (ou seja,  $\overline{\textsc{TAUT}}$) está em NP.
O problema $\overline{\textsc{TAUT}}$ consiste nas fórmulas booleanas que não são tautologias.
Para verificar que uma fórmula não é uma tautologia, basta encontrar uma atribuição que a falsifique. Existem algoritmos que fazem essa verificação \textbf{em tempo polinomial}.
Logo, $\overline{\textsc{TAUT}}$ está em NP, o que implica que \textsc{TAUT} está em coNP.

\newpage

\section*{Exercício 6}
% enunciado
Seja $G = (V,E)$ um grafo. Um conjunto $S \subseteq V$ é \textit{independente} se quaisquer dois vértices de $S$ não são adjacentes. Ou seja, não há nenhuma aresta do grafo com as duas pontas em $S$. O problema IS consiste no seguinte: dado um grafo $G$ e um inteiro $k\geq 0$, existe um conjunto indepentente em $G$ com $k$ vértices? Mostre que o problema IS é NP-completo. \textbf{Não} pode usar o fato que o problema \textsc{clique} é NP-completo, mas pode se inspirar na prova deste teorema.

\bigskip

% resposta
% https://youtu.be/GCLGc5lh0_w
\subsection*{Resposta:}
Vamos mostrar que o problema \textsc{IS} é tanto NP quanto o problema NP-completo \textsc{3-SAT} pode ser reduzido para \textsc{IS}.

\bigskip

Primeiramente, vamos provar que o problema \textsc{IS} está em NP apresentando um algoritmo polinomial que, dado um grafo $G=(V,E)$ e um conjunto de vértices $S$ (um certificado), verifica se $S$ é independente.
O algoritmo consiste em verificar se existe aresta para cada par de vértices em $S$.
Se não existir, então $S$ é independente.
Caso contrário, $S$ não é independente.
A complexidade desse algoritmo é $O(|V|+|E|)$, polinomial.

\bigskip

Agora, vamos provar que $\textsc{3-SAT} \leq_P \textsc{IS}$.
Seja $\phi$ uma fórmula booleana em \textbf{3-CNF} com $n$ variáveis e exatamente $k$ cláusulas.
Considere um grafo $G=(V,E)$.
Cada vértice de $G$ representa um literal de cada cláusula de $\phi$.
Vamos criar as arestas de forma que
\begin{itemize}
  \item os 3 vértices de literais de cada cláusula sejam ligados 2 a 2, formando um triângulo - é necessário que exatamente 1 literal de cada cláusula seja verdadeiro para garantir a independência do conjunto;
  \item os vértices de literais complementares sejam ligados (ou seja, todos os $x_i$ e $\neg x_i$) - isso garante que $x_i$ e $\neg x_i$ nunca serão simultaneamente verdadeiros ao escolher um conjunto independente.
\end{itemize}

\medskip

Agora vamos provar que $\phi$ é satisfatível $\iff$ $G$ tem um conjunto independente com $k$ vértices.

($\Rightarrow$) Seja $\alpha$ uma atribuição que torna $\phi$ verdadeira.
Logo, todas as cláusulas são verdadeiras.
Como temos exatamente $k$ cláusulas e queremos montar um conjunto independente $S$ com exatamente $k$ vértices, então exatamente 1 literal em cada cláusula é verdadeiro para que todas as cláusulas sejam verdadeiras.
Pelas propriedades citadas anteriormente, não há conflito de literais complementares.
Logo, $S$ é um conjunto independente.

($\Leftarrow$) Seja $S$ um conjunto independente com $k$ vértices.
Como $S$ é independente, não há conflito de literais complementares.
Como $S$ tem $k$ vértices e temos $k$ cláusulas em $\phi$, então exatamente 1 literal em cada cláusula é verdadeiro para que todas as cláusulas sejam verdadeiras.
Como $S$ é independente, então todas as cláusulas são verdadeiras.
Logo, $\phi$ é satisfatível.

\bigskip

Logo, $\textsc{3-SAT} \leq_P \textsc{IS}$, o que implica que \textsc{IS} é NP-completo.

\newpage

\section*{Exercício 9}
% enunciado
Mostre que o problema abaixo é NP-completo.
\begin{center}
  \begin{minipage}{33em}
    \textbf{Problema} \textsc{Mochila:} Dado um número $W$, um número $V$, um número inteiro positivo $n$, uma coleção de números $w_1, \dots, w_n$, e uma coleção de números $v_1, \dots, v_n$, decidir se existe um subconjunto $S$ de $\{1, \dots, n\}$ tal que
    \[\sum_{i \in S}w_i \leq W \mbox{ e } \sum_{i \in S}v_i \leq V .\]
  \end{minipage}
\end{center}
Pode assumir que o problema \textsc{partição} (Exercício 8) é NP-completo.

\bigskip

% resposta
\subsection*{Resposta:}
\end{document}