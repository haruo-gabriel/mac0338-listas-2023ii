\documentclass{article}

\title{MAC0338 - Entrega da lista 3}
\author{}
\date{}

\usepackage[a4paper, margin=1.25cm]{geometry}
\usepackage{amsmath, amssymb, amsthm}
\usepackage{bm}
\usepackage[skip=2pt]{parskip}    
\usepackage{wrapfig}
\usepackage{braket}
\usepackage{mathtools}
\usepackage[noend, linesnumbered]{algorithm2e}
\SetKwComment{Comment}{$\triangleright$\ }{}
\RestyleAlgo{ruled}

\newcommand{\N}{\mathbb{N}}
\newcommand{\bigO}[1]{\ensuremath{\mathcal{O}(#1)}}

\begin{document}

\maketitle

\section*{Exercício 1}
% enunciado

\underline{\textit{Qual é o número esperado de comparações executadas na linha 6 do algoritmo?}}

Seja $X$ uma v.a. que represente o número total de execuções da linha 6.
Seja $X_i = \begin{cases*}1 \mbox{ se a linha 6 é executada}\\0 \mbox{ caso contrário}\end{cases*}$.

Logo $X = X_1 + \dots + X_n$.

Note que a linha 6 é executada apenas quando a condição da linha 4 (se $v[i] >$ maior) é falsa.
Portanto, seja a esperança de $X_i$ definida como
\begin{align*}
  E[X_i] &= \mbox{prob. de que } A[i] \mbox{ NÃO seja o máximo em } A[1..i] \\
  &= 1 - \mbox{prob. de que } A[i] \mbox{ seja o máximo em } A[1..i] \\
  &= 1 - \dfrac{1}{i} \\
  &= \dfrac{i-1}{i}.
\end{align*}
Logo, o número esperado de execuções da linha 6, ou seja, a esperança de $X$ é
\begin{align*}
  E[X] &= E[X_1 + \dots + X_n] = E[X_1] + \dots + E[X_n] \\
  &= \frac{1-1}{1} + \dots + \frac{n-1}{n} = 0 + \frac{1}{2} + \frac{2}{3} + \frac{3}{4} + \dots + \frac{n-1}{n} \\
  &= \sum_{k=1}^{n} \frac{k-1}{k}
\end{align*}
Além disso, note que
\begin{align*}
  E[X] = \sum_{k=1}^{n} \frac{k-1}{k} < \sum_{k=1}^{n} 1 = n
\end{align*}
E portanto, $E[X]$ é $\bigO{n}$.

\bigskip

\underline{\textit{Qual é o número esperado de atribuições efetuadas na linha 7 do algoritmo?}}

Seja $Y$ uma v.a. que represente o número total de execuções da linha 7.
Seja $Y_i = \begin{cases*}1 \mbox{ se a linha 7 é executada}\\0 \mbox{ caso contrário}\end{cases*}$.

Logo $Y = Y_1 + \dots + Y_n$.

Note que a linha 7 é executada apenas quando a condição da linha 6 (se $v[i] <$ menor) é falsa.
Portanto, seja a esperança de $Y_i$ definida como
\begin{align*}
  E[Y_i] = \mbox{prob. de que } A[i] \mbox{ seja o mínimo em } A[1..i] = \dfrac{1}{i} \\
\end{align*}
Logo, o número esperado de execuções da linha 7, ou seja, a esperança de $Y$ é
\begin{align*}
  E[Y] = E[Y_1 + \dots + Y_n] = E[Y_1] + \dots + E[Y_N] = \frac{1}{1} + \dots + \frac{1}{n} = \sum_{k=1}^{n} \frac{1}{n}
\end{align*}
Além disso, note que
\begin{align*}
  E[Y] = \sum_{k=1}^{n} \frac{1}{k} < 1 + \lg n
\end{align*}
E portanto, $E[Y]$ é $\bigO{\lg_n}$.

\newpage

\section*{Exercício 3}
% enunciado

% explicitar o algoritmo
\begin{algorithm}
  \DontPrintSemicolon
  \caption{PARTICIONA-MEDIANA}
  \KwData{A, p, r}
  \KwResult{A particionado de p até r}
  \SetKwProg{Fn}{Function}{:}{end}
  \Fn{PARTICIONA-MEDIANA($A, p, r$)}{
    $m \gets$ \texttt{mediana}($A, p, r$) \Comment*[r]{O(n)}
    $k \gets$ índice de $m$ no vetor A  \Comment*[r]{O(n)}
    $A[k] \leftrightarrow A[r]$\;
    \Return \texttt{particiona}($A, p, r$) \Comment*[r]{O(n)}
  }
\end{algorithm}

\begin{algorithm}
  \DontPrintSemicolon
  \caption{SELECT-MEDIANA}
  \KwData{$A, p, r, k$}
  \KwResult{$k$-ésimo menor elemento de $A$}
  \SetKwProg{Fn}{Function}{:}{end}
  \Fn{SELECT-MEDIANA($A, p, r, k$)}{
    \lIf{$p = r$}{\Return $A[p]$ }
    $q \gets$ \texttt{particiona-mediana}($A, p, r$) \Comment*[r]{3*O(n)}
    \lIf{$k = q - p + 1$}{\Return $A[q]$}
    \eIf{$k < q - p + 1$}{
      \Return \texttt{select-mediana}($A, p, q-1, k$) \Comment*[r]{T(n/2)}
    }{
      \Return \texttt{select-mediana}($A, q+1, r, k-(q-p+1)$) \Comment*[r]{T(n/2)}
    }
  }
\end{algorithm}

% explicar o algoritmo

Note que os algoritmos são semelhantes aos \texttt{PARTICIONA-ALEA} e \texttt{SELECT-ALEA}, sendo este \bigO{n}.
Nesse algoritmo, o lugar certo do pivô \textbf{sempre} será a posição $\lfloor (p+r)/2 \rfloor$, ou seja, dividirá o vetor (ou subvetor, em uma chamada recursiva) em 2 segmentos descritos a seguir.

Como temos esses invariantes no final de cada execução de \texttt{PARTICIONA}($A,p,r$)
\begin{itemize}
  \item todos os elementos do segmento à esquerda do pivô são menores que ele 
  \item todos os elementos do segmento à direita do pivô são maiores que ele
\end{itemize}
então se $k$ for
\begin{itemize}
  \item igual a $\lfloor (p+r)/2 \rfloor$, temos o que queremos
  \item menor que $\lfloor (p+r)/2 \rfloor$, com certeza o $k$-ésimo menor elemento está no segmento à esquerda do pivô
  \item maior que $\lfloor (p+r)/2 \rfloor$, com certeza o $k$-ésimo menor elemento está no segmento à direita do pivô.
\end{itemize}

%\textit{obs.: o algoritmo se assemelha muito à busca binária, mas funciona em um vetor não necessariamente ordenado}

% provar o consumo de tempo linear

Para demonstrar o consumo de tempo, note que \texttt{SELECT-MEDIANA} é um algoritmo recursivo.
Portanto, vamos construir uma fórmula de recorrência $T(n)$ para essa função:
\begin{align*}
  T(n) &= T(n/2) + 3n
  = \Bigl(T(n/4) + \frac{3}{2}n\Bigr) + 3n
  = \Bigl(T(n/8) + \frac{3}{4}n\Bigr) + \Bigl(3 + \frac{3}{2}\Bigr) n \\
  \vdots \\
  &= T(n/2^q) + \sum_{i=0}^{q-1} \frac{3}{2^i}n
  = T(n/2^q) + 3n \sum_{i=1}^{q} \frac{1}{2^{i-1}}
  = T(n/2^q) + 3n \underbrace{\frac{1 - (1/2)^q}{1 - 1/2}}_{\mbox{soma de PG finita}} \\
  &= T(n/2^q) + 3n \Bigl(2 -\frac{1}{2^{q-1}} \Bigr)
  = T(n/2^q) + 3n \Bigl(2 -\frac{2}{2^q} \Bigr)
\end{align*}
Para $2^q = n$ e $n$ potência de 2, temos a expressão
\begin{align*}
  T(n) = T(1) + 3n \Bigl(2 - \frac{2}{n} \Bigr) = 1 + 6n - 6 = 6n - 5
\end{align*}
Vamos provar por indução em $q$, $q=\lg n$, que $T(2^q) = 6\cdot 2^q - 5$.

Para $q=0$,
\begin{align*}
  T(2^0) = 6 \cdot 2^0 - 5 \implies T(1) = 1
\end{align*}

Para $q \geq 1$, suponha $T(2^{q-1}) = 6 \cdot 2^{q-1} - 5$, então
\begin{align*}
  T(2^q) &= T(2^{q-1}) + 3 \cdot 2^q
  = (6 \cdot 2^{q-1} - 5) + 3 \cdot 2^q
  = 3 \cdot 2^q - 5 + 3 \cdot 2^q
  = 6 \cdot 2^q - 5 ,
\end{align*}
como queríamos provar.

Logo, para $T(n) = 6n - 5$, o consumo de tempo é proporcional a \bigO{n}.

\newpage

\section*{Exercício 9}
% enunciado

\begin{algorithm}
  \DontPrintSemicolon
  \caption{MEDIANA2VETORES}
  \KwData{$X, p, q, Y, r, s$}
  \KwResult{mediana do vetor combinado $X$ com $Y$}
  \SetKwProg{Fn}{Function}{:}{end}
  \Fn{MEDIANA($X, p, q, Y, r, s$)}{
    \If{$p=r$ \textbf{and} $r=s$}{
      \eIf{$X[p] < Y[r]$}{\Return $X[p]$}{\Return $Y[r]$}
    }
    $x \gets \lfloor\frac{p + q}{2}\rfloor$ \;
    $y \gets \lfloor\frac{r + s}{2}\rfloor$ \;
    \If{$X[x] < Y[y]$}{
      \Return \texttt{MEDIANA2VETORES}($X, x+1, q, Y, r, y$) \Comment*[r]{T(n/2)}
    }
    \Else{ 
      \Return \texttt{MEDIANA2VETORES}($X, p, x, Y, y, s$) \Comment*[r]{T(n/2)}
    }
  }
\end{algorithm}

O algoritmo acima recebe 2 vetores $X,Y$, ambos de tamanho $n$, ordenados e \textbf{sem elementos repetidos}.
Os índices $p,q$ são referentes ao subvetor de $X$ que será processado, enquanto $r,s$ são referentes ao subvetor de $Y$.
É retornado a mediana de um vetor que combinaria todos os elementos de $X$ e $Y$.

O algoritmo funciona recebendo inicialmente \texttt{MEDIANA2VETORES}($X,1,n,Y,1,n$) e
\begin{enumerate}
  \item encontra o índice da mediana de cada vetor, sendo $x$ o índice da mediana de $X[1..n]$ e $y$, o de $Y[1..n]$.
  \item Se $p = q$ e $r = s$, retorna o mínimo entre $X[p]$ e $Y[r]$.
  \item Se $p \neq q$ e $r \neq s$, então quantifica $X[x]$ e $Y[y]$.
  \begin{enumerate}
    \item Se $X[x] < Y[y]$, então faz a recursão considerando $X[x+1..q]$ e $Y[r..y]$
    \item Senão (\textit{se $X[x] < Y[y]$}), então faz a recursão considerando $X[p..x]$ e $Y[y..s]$
  \end{enumerate}
\end{enumerate}

\bigskip

A análise de consumo de tempo considera a fórmula recursiva $T(n) = T(n/2) + \bigO{1}$, sendo $T(n/2)$ a chamada recursiva e $\bigO{1} = 1$ as chamadas das outras linhas.
\begin{align*}
  T(n) &= T(n/2) + \bigO{1} = (T(n/4) + \bigO{1}) + \bigO{1} = \dots = T(n/2^k) + k\bigO{1} = T(n/2^k) + k
\end{align*}
Considerando $k = \lg n$ e $n$ potência de 2, temos a expressão $T(n) = T(1) + \lg n = 1 + \lg n$.

Vamos provar por indução em $k$, que $T(2^k) = 1 + k$.

Para $k = 0$, diretamente $$T(2^0) = 1 + 0 \implies T(1) = 1$$.

Para $k \geq 1$ e supondo $T(2^{k-1}) = 1 + (k-1)$, temos
$$
  T(2^k) = T(2^{k-1}) + \bigO{1} = 1 + (k-1) + 1 \implies T(2^k) = k + 1,
$$
como queríamos provar.

Logo, o consumo de tempo da função \texttt{MEDIANA2VETORES} é $1 + \lg n$, proporcional a \bigO{\lg n}.

\end{document}